\input slimac

\newdimen\rmargin \rmargin=\fontquad
\def\demo{\vskip\parskip\begingroup
  \catcode`\$=12 \catcode`\&=12 \afterassignment\ddemo\count255=}
\def\ddemo#1{\line{\setbox0=\hbox{\epsfbox{pics/talk2-\the\count255.eps}}%
  \vbox to\ht0{\vss\hbox{\strut\vsf #1}\vss}\hfil\box0\hskip\rmargin}
  \endgroup}

\typeface{vsf}{lstr}

\blankslide

{\baselineskip=2\grid\twentysfb
\incolour\headingcolour{Geomlab:\hfil\break
\null\quad Exploring Computer Science}\par}
\vfill
{\twelvepoint\sf\noindent
Mike Spivey\par
\noindent
\sfi Computer Science Tutor\hfil\break
Oriel College, Oxford}

%> define digits = explode("0123456789");

%> define select(0, x:xs) = x
%>   | select(n+1, x:xs) = select(n, xs);

%> define digit(d) = select(d, digits);

%> define num(0) = "0"
%>   | num(n) = 
%>       (let h(0) = "" | h(r) = h(r div 10) ^ digit(r mod 10) in h(n));

%> define picture(n, p, r) = 
%>   epswrite(p, "pics/talk2-" ^ num(n) ^ ".eps", r, 0.95);

%> define pic(n, p) = picture(n, p, 36);
%> define medpic(n, p) = picture(n, p, 72);
%> define bigpic(n, p) = picture(n, p, 108);

% Picture of specified height.
%> define htpic(n, p) = picture(n, p, 72*sqrt(aspect(p)));

\slide{What's programming like?}

The old-fashioned view:

\item A program is a list of commands.
\item Computer memory is boxes that contain numbers.
\item Algorithms are designed with flowcharts.
\item It's all ones and zeros!

\slide{What's it really like?}

A modern view:

\item A program is an artificial world.
\item Objects behave according to precise rules.
\item By combining simple behaviours, a complex whole emerges.
\item We can use things without knowing how they are made.

\slide{The algebra of pictures}

\begingroup
  \divide\hsize by2 \advance\hsize by-\fontquad
  \valign{&\vss#\vss\cr
    \demo1{man}&\demo3{man $ woman}\cr
    \noalign{\qquad}
    \demo2{woman}&\demo4{man & man}\cr}
\endgroup

%> pic(1, man);
%> pic(2, woman);
%> pic(3, man $ woman);
%> pic(4, man & man);

\slide{The algebra of pictures}

\demo5{man $ (woman & tree)}
\demo6{(man $ woman) & tree}

%> pic(5, man $ (woman & tree));
%> pic(6, (man $ woman) & tree);

\slide{Defining functions}

\nomargintrue
\verbatim
define f(x) = x $ (x & x)
\endverb

\demo7{f(man)}
\demo8{f(woman)}
\demo9{f(man $ tree)}

%> define f(x) = x $ (x & x);
%> pic(7, f(man));
%> pic(8, f(woman));
%> pic(9, f(man $ tree));

\slide{Composing functions}

\demo10{f(f(man))}
\demo101{f(f(f(man)))}

%> medpic(10, f(f(man)));
%> medpic(101, let p = f(f(f(man))) in _blank(aspect(p)));

\slide{Composing functions}

\demo10{f(f(man))}
\demo11{f(f(f(man)))}

%> medpic(11, f(f(f(man))));

\slide{Recurrence and recursion}

\nomargintrue
\verbatim
define row(1, p) = p
  | row(n, p) = p $ row(n-1, p) when n > 1
\endverb

%> define row(1, p) = p
%>   | row(n, p) = p $ row(n-1, p) when n > 1;

\demo12{row(4, man)}
\demo13{row(6, man)}
\demo14{row(10, man)}

%> pic(12, row(4, man));
%> pic(13, row(6, man));
%> pic(14, row(10, man));


\slide{Other recursive patterns}

\nomargintrue
\verbatim
define spiral(1) = bend
  | spiral(n) = 
      arm(n) & (rot(arm(n-1)) 
                  $ rot2(spiral(n-1))) when n > 1
\endverb

%> define spiral(1) = bend
%>  | spiral(n) =
%>      arm(n) & (rot(arm(n-1)) $ rot2(spiral(n-1))) when n > 1;

\vfil
\line{\hfill\epsfbox{pics/talk2-15.eps}\hfill}

%> medpic(15, special2);

\slide{The results}

\demo16{spiral(10)}
\demo17{zagzig(10)}

%> medpic(16, spiral(10));
%> medpic(17, zagzig(10));

\slide{The big challenge}

Suppose we have a tile that fits next to itself at half scale:

\demo18{(T & blank) $ T}

%> htpic(18, (T & blank) $ T);

\slide{The big challenge}

Suppose we have a tile that fits next to itself at half scale:

\demo19{(T & blank) $ T}

%> htpic(19, colour((T & blank) $ T));

\slide{The big challenge}

\dots and another copy fits here:

\demo20{(T & rot(T)) $ T}

%> htpic(20, colour((T & rot(T)) $ T));

\slide{The big challenge}

Then we can add another layer:

\demo21{side(2) $ T}

%> htpic(21, colour(side(2) $ T));

\slide{The big challenge}

\dots and arrange four copies around a central square:

\rmargin=0pt
\demo22{frame(blank, side(2), U)}

%> bigpic(22, colour(frame(blank, side(2), U)));

\slide{The big challenge}

Finally, find something to fill in the corners:

\demo23{frame(corner(2), side(2), U)}

%> bigpic(23, colour(frame(corner(2), side(2), U)));

(A picture that Escher drew, and another that he didn't.)

\slide{Functional programming}

\item No `programming' variables -- use mathematical variables instead.
\item No loop commands -- use recursion instead

Two benefits:

\item Easier to make programs from independent pieces.
\item Easier to reason about them mathematically.

\slide{Abstract data types}

A set of values with operations defined on them.

\item The operations obey rules that can be specified algebraically.
\item The computer representation of values is hidden.

\slide{Abstract data types}

Example: pictures with \verb/$/, \verb/&/, \verb/rot/.

\item {\vsf rot($p$ \$ $q$) = rot($q$) \& rot($q$)}.
\item {\vsf aspect(rot($p$)) = 1/aspect($p$)}.

Pictures are represented using coordinates and transformation matrices
in a way that is hidden.

\slide{Enriching school mathematics}

\item There's more to algebra than ``$x$ is the unknown''.
\item There's more to recurrences than ``guess the next number in the
series''.
\item May we dream of a world where survival depends on mathematics?

\slide{The GeomLab site}

\verb@http://web.comlab.ox.ac.uk/geomlab@

The software:
\item runs from the web page.
\item Java-based -- no installation required.

Teaching materials:
\item full set of worksheets for a 1--2 day activity.

\bye



 
