\input slimac

\newdimen\picwidth \picwidth=108truept

\def\picture#1{\hbox{\epsfbox{pics/oview-#1.eps}}}

\def\demo{\vskip\parskip
  \begingroup
    \afterassignment\demoi\count255=}
\def\demoi{\catcode`\$=12 \catcode`\&=12 \demoii}
\def\demoii#1{\line{\setbox0=\picture{\the\count255}%
%  \vbox to\ht0{\vss \hbox{\strut\vsf #1} \vss}%
   \vbox to\ht0{\vss \halign{\strut\vsf ##\cr#1\crcr} \vss}%
    \hfil
    \hbox to\picwidth{%
      \hskip 0pt plus 1fil minus 1fil
      \box0
      \hskip 0pt plus 1fil}}
  \endgroup}

%> define digits = explode("0123456789");
%>
%> define select(0, x:xs) = x
%>   | select(n+1, x:xs) = select(n, xs);
%>
%> define digit(d) = select(d, digits);
%>
%> define num(n) = digit(n) when n < 10
%>   | num(n) = num(n div 10) ^ digit(n mod 10);
%>
%> define picture(n, p, r) = 
%>   epswrite(p, "pics/oview-" ^ num(n) ^ ".eps", r, 0, 0.95);
%>
%> define pic(n, p) = picture(n, p, 36);
%> define medpic(n, p) = picture(n, p, 72);
%> define bigpic(n, p) = picture(n, p, 108);
%>
%> { Picture of specified height. }
%> define htpic(n, p) = picture(n, p, 72*sqrt(aspect(p)));


%%%%%%%%%%%%%%%%

\blankslide

{\baselineskip=2\grid\twentysfb
\incolour\headingcolour{GeomLab: Exploring\hfil\break\null\qquad
Computer Science}\par}
\vfill
{\twelvepoint\sf\noindent
Michael Spivey\par
\noindent
\sfi Computer Science Tutor\hfil\break
Oriel College, Oxford}

\slide{What's computer programming like?}

The old-fashioned view:

\item A program is a list of commands.
\item Computer memory is boxes that contain numbers.
\item Algorithms are designed with flowcharts.
\item It's all ones and zeros!

\slide{What's it really like?}

A modern view:

\item A program is an artificial world.
\item Objects behave according to precise rules.
\item By combining simple behaviours, a complex whole emerges.
\item We can use things without thinking about how they are made.

\slide{A world of pictures}

\begingroup
  \divide\hsize by2 \advance\hsize by-\fontquad
  \valign{&\vss#\vss\cr
    \demo1{man}&\demo3{man $ woman}\cr
    \noalign{\qquad}
    \demo2{woman}&\demo4{man & man}\cr}
\endgroup

%> pic(1, man);
%> pic(2, woman);
%> pic(3, man $ woman);
%> pic(4, man & man);

\slide{The algebra of pictures}

\demo5{man $ (woman & tree)}
\demo6{(man $ woman) & tree}

%> pic(5, man $ (woman & tree));
%> pic(6, (man $ woman) & tree);

\slide{Defining functions}

\verb/define f(x) = x $ (x & x)/

\demo7{f(man)}
\demo8{f(woman)}
\demo9{f(man $ tree)}

%> define f(x) = x $ (x & x);
%> pic(7, f(man));
%> pic(8, f(woman));
%> pic(9, f(man $ tree));

\slide{Composing functions}

\demo10{f(f(man))}
\demo101{f(f(f(man)))}

%> medpic(10, f(f(man)));
%> medpic(101, let p = f(f(f(man))) in _blank(aspect(p)));

\slide{Composing functions}

\demo10{f(f(man))}
\demo11{f(f(f(man)))}

%> medpic(11, f(f(f(man))));

\slide{Using recursion}

Suppose we want to draw a row of trees:
\medskip
\picture{12}
We could write \verb/tree $ tree $ tree $ ... $ tree/.

But is there a better way?

%> define trees(1) = tree | trees(n+1) = trees(n) $ tree;

%> bigpic(12, trees(10));

\slide{Using recursion}

\item A row of one tree is just a tree: \verb/trees(1) = tree/

\item An a row of \verb/n+1/ trees is a row of \verb/n/ trees with an
extra tree at the end: \verb/trees(n+1) = trees(n) $ tree/.

So we write:
\verbatim
define trees(1) = tree
  | trees(n+1) = trees(n) $ tree
\endverb

\slide{Unwinding the recursion}

Now we (or the computer) can calculate:
\verbatim
trees(5) = trees(4) $ tree
  = trees(3) $ tree $ tree
  = trees(2) $ tree $ tree $ tree
  = trees(1) $ tree $ tree $ tree $ tree
  = tree $ tree $ tree $ tree $ tree
\endverb
\medskip
\centerline{\picture{13}}

%> bigpic(13, trees(5));

\slide{Many trees make a forest}

%> define forest(1, c) = trees(c)
%>   | forest(r+1, c) = forest(r, c+1) & trees(c);

%> bigpic(14, forest(10, 10));

How can we draw this picture?
\medskip
\centerline{\picture{14}}

\slide{Recursion again}

Let's write \verb/forest(r, c)/ for a forest with \verb/r/ rows,
and \verb/c/ trees in the bottom row.

\item First, we see that \verb/forest(1, c) = trees(c)/.

\item Second, \verb/forest(r+1, c)/ has a bottom row that is
\verb/trees(c)/, and the rest is \verb/forest(r, c+1)/.

So we write
\verbatim
define forest(1, c) = trees(c)
  | forest(r+1, c) = 
      forest(r, c+1) & trees(c)
\endverb

\slide{As big as you like}

\demo15{forest(2, 2)}
\demo17{forest(20, 20)}

%> pic(15, forest(2, 2));
%> pic(16, forest(4, 6));
%> medpic(17, forest(20, 20));

\slide{More recursive patterns}

Using \picture{20} and \picture{21}, we can making tiling patterns:
\bigskip
\centerline{\picture{22}\hfil\picture{23}}
\medskip
These too can be described by short programs.

%> picture(20, bend, 18);
%> picture(21, straight, 18);
%> medpic(22, spiral(10));
%> medpic(23, zagzig(10));

\slide{The big challenge}

Suppose we have a tile that fits next to itself at half scale:

\demo118{(T & blank) $ T}

%> htpic(118, (T & blank) $ T);

\slide{The big challenge}

Suppose we have a tile that fits next to itself at half scale:

\demo119{(T & blank) $ T}

%> htpic(119, colour((T & blank) $ T));

\slide{The big challenge}

\dots and another copy fits here:

\demo120{(T & rot(T)) $ T}

%> htpic(120, colour((T & rot(T)) $ T));

\slide{The big challenge}

Then we can add another layer:

\demo121{side(2) $ T}

%> htpic(121, colour(side(2) $ T));

\slide{The big challenge}

\dots and arrange four copies around a central square:

\demo122{frame(blank, side(2), U)}

%> bigpic(122, colour(frame(blank, side(2), U)));

\slide{The big challenge}

Finally, find something to fill in the corners:

\demo123{frame(corner(2),\cr\qquad side(2), U)}

%> bigpic(123, colour(frame(corner(2), side(2), U)));

(A picture that Escher drew, and another that he didn't.)

\slide{Functional programming}

\item No `programming' variables -- use mathematical variables instead.
\item No loop commands -- use recursion instead

Two benefits:

\item Easier to make programs from independent pieces.
\item Easier to reason about them mathematically.

\slide{Abstract data types}

A set of values with operations defined on them.

\item The operations obey rules that can be specified algebraically.
\item The computer representation of values is hidden.

\slide{Abstract data types}

Example: pictures with \verb/$/, \verb/&/, \verb/rot/.

\item {\vsf rot($p$ \$ $q$) = rot($q$) \& rot($p$)}.
\item {\vsf aspect(rot($p$)) = 1/aspect($p$)}.

Pictures are represented using coordinates and transformation matrices
in a way that is hidden.

\slide{Enriching school mathematics}

\item There's more to algebra than ``$x$ is the unknown''.
\item There's more to recurrences than ``guess the next number in the
series''.
\item May we dream of a world where survival depends on mathematics?

\slide{The GeomLab site}

\verb@http://web.comlab.ox.ac.uk/geomlab@

The software:
\item runs from the web page.
\item Java-based -- no installation required.

Teaching materials:
\item full set of worksheets for a 1--2 day activity.

\bye
