%
% Hypertext bookmarks and links
%

\catcode`\@=11
{\catcode`\^=7 \global\catcode`\^^I=10} % TAB characters are spaces

\ifpdf

  \pdflinkmargin=1pt % Border of the box around a hyperlink

  % A boxed link
  \def\hyperbox#1#2#3{% attr action text
    \pdfstartlink attr {#1} #2\strut#3\pdfendlink}

  % A link that leads to an anchor in the same document
  \def\hyperlink#1{% anchor text
    \hyperbox{/C [1 0 0] /Border [0 0 1]}{goto name {#1}}}

  \def\hyperurl#1{% url text
    \hyperbox{/C [0 1 0] /Border [0 0 1]}%
      {user {/A << /Type /Action /S /URI /URI (#1) >>
        /Subtype /Link}}}

  % An entry in the table of contents
  \def\hyperoutline#1#2#3{% childcount anchor title
    \pdfoutline goto name {#2} count #1 {#3}}

  % An anchor
  \def\hyperanchor#1{\pdfdest name {#1} xyz\relax}

\else

  % Postcript code used for hypertext
  \preamble{\special{! 
    /hyperdict 30 dict def		% working storage
    /DvipsToPDF { % Convert dvips units to PDF units
      72 mul Resolution div } def
    /TeXToDvips { % Convert TeX units to dvips units
      72.27 div Resolution mul } def
    /HTAnchor { % Anchor: name
      hyperdict begin
      cvn /anch exch def		% anch := name
      % Find point 1/2in above currentpoint
      currentpoint 			% [x y]
      exch DvipsToPDF /hoff exch def	% [y], hoff := x'
      DvipsToPDF % 36 sub 		% [y'-1/2in]
      vsize 72 sub exch sub 		% transform to PDF axes
      /voff exch def 			% voff := y''
      % Make pdfmark, with PDF page coordinates
      mark /View [ /XYZ hoff voff null ] /Dest anch /DEST pdfmark 
      end } def
    /HTOutline { % Bookmark: childcount, anchor, title
      hyperdict begin
      /title exch def			% title := title
      cvn /anch exch def		% anch := anchor
      /count exch def			% count := childcount
      mark /Count count /Dest anch /Title title /OUT pdfmark 
      end} def
    /HTBegin { % Start of hyperlink
      hyperdict begin 
      currentpoint pop /llx exch def 	% llx := x
      end } def	
    /HTRect { % Calculate rectangle: height, depth
      currentpoint 			% [ht dp x y]
      exch /urx exch def		% [ht dp y], urx := x
      exch 1 index			% [ht y dp y]
      exch TeXToDvips add 		% [ht y y+dp']
      /lly exch def 			% [ht y], lly := y+dp'
      exch TeXToDvips sub 		% [y-ht']
      /ury exch def 			% ury := y-ht'
      /rect [ llx lly urx ury ] def } def
    /HTLink { % Complete hyperlink: height, depth, anchor
      hyperdict begin
      cvn /anch exch def 		% anch := anchor
      HTRect
      % Make pdfmark, with user coordinates
      mark /Subtype /Link /Dest anch 
	/Border [0 0 1 TeXToDvips] /Color [1 0 0] /Rect rect /ANN pdfmark 
      end } def
    /HTLaunch { % Link to external object: height depth type dest
      hyperdict begin
      /dest exch def
      /type exch def
      HTRect
      mark /Subtype /Link /Action /Launch type dest
	/Border [0 0 1 TeXToDvips] /Color [0 1 0] /Rect rect /ANN pdfmark
      end } def
    systemdict /pdfmark known not 
      { userdict /pdfmark systemdict /cleartomark get put } if }}

  \def\pdfspecial#1{\special{ps:SDict begin #1 end}}

  \def\hyperbox#1#2{% code text
    \pdfspecial{HTBegin}#2\pdfspecial{\strip@pt{\ht\strutbox}\space 
      \strip@pt{\dp\strutbox}\space #1}}

  \def\hyperurl#1{% url text
    \hyperbox{/URI (#1) HTLaunch}}

  \def\hyperlink#1{% anchor text
    \hyperbox{(#1) HTLink}}

  \def\hyperoutline#1#2#3{% childcount anchor title
    \pdfspecial{#1 (#2) (#3) HTOutline}}

  \def\hyperanchor#1{\pdfspecial{(#1) HTAnchor}}

\fi

% This strips the 'pt' from a dimen or glue.
% It removes trailing plus or minus <dimen> too:
\def\strip@pt#1{\expandafter\@xyzzy\the#1!}
{\catcode`p=12 \catcode`t=12 \gdef\@xyzzy#1pt#2!{#1}}

\newcount\gensym

\def\hyperanchor@#1#2{% cs tag
  \global\advance\gensym by1 \xdef#1{#2.\the\gensym}%
  \hyperanchor#1}


% Bookmarks

% Each bookmark has an anchor in the text, and a definition that's
% output on the .bko file.  The definitions need to be post-processed
% to escape special characters and remove formatting; the result
% becomes the .bkm file that is input on the next run.

% Output file for bookmarks
\newwrite\bkmout
\newif\ifbkmopen \bkmopenfalse
\def\checkbkm{\ifbkmopen\else
  \immediate\openout\bkmout=\jobname.bko
  \immediate\write\bkmout{\relax}%
  \global\bkmopentrue\fi}

\def\bookmark#1#2#3{% level mark title
  {\enableprotectii
    \edef\next{\write\bkmout{\string\bookmarkdef{#1}{#2}{#3}}}%
    \checkbkm \next}%
  \hyperanchor{#2}%
  \gdef\bmark{#2}}

\def\bookmark@#1#2#3{% level mark title
  \global\advance\gensym by1
  \bookmark{#1}{#2.\the\gensym}{#3}}

% We need to count the children of each bookmark, so we make two
% passes over the bookmark file, saving the material in \bmarktoks temporarily
\newtoks\bmarktoks
\def\addtobmarks#1{\bmarktoks=\expandafter{\the\bmarktoks#1}}

\newcount\parent

% \C@i is the current bookmark for each level i
% \B@m is the number of children of each bookmark m
\def\bookmarkdef#1#2#3{% level mark title
  \addtobmarks{\bookm@rkdef{#2}{#3}}           % Save for later
  \expandafter\def\csname C@#1\endcsname{#2}   % current[level] = mark
  \expandafter\def\csname B@#2\endcsname{0}    % children[mark] = 0
  \ifnum#1>1 \parent=#1 \advance\parent by-1   % parent = level-1
    \expandafter\let\expandafter\next          % next = current[parent]
      \csname C@\the\parent\endcsname
    \ifx\next\relax\else
      \count@=\csname B@\next\endcsname        % childen[next]++
      \advance\count@ by1
      \expandafter\edef\csname B@\next\endcsname{\the\count@}
  \fi\fi}

\def\bookm@rkdef#1#2{%\def\next{\csname B@#1\endcsname}%
  \count@=-\csname B@#1\endcsname \hyperoutline{\the\count@}{#1}{#2}}

\ifreadable{\jobname.bkm}%
  \expandafter\input \jobname.bkm % input the bookmarks
  \the\bmarktoks \bmarktoks={}% second pass over the bookmarks
\fi

{\catcode`\^=7 \global\catcode`\^^I=15} % TAB characters are illegal
\catcode`\@=12
