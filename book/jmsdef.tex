% jmsdef.tex 
% $Id: jmsdef.tex 2 2006-06-12 09:26:54Z mike $

% Basic definitions for JMS format (also used in MetaPost)

\catcode`@=11

\let\protect=\relax

%
% Font loading
%

% Six font schemes:
%
%   Scheme	Body font	\iflucida	CM fonts
%   0 (default)	lbr		true
%   1		lsr		true
%   2		cmr		false		sauter
%   3		cmss		false		sauter
%   4		cmr		false		basic
%   5		cmss		false		basic
%
% Schemes 4 and 5 are like 2 and 3, but use just the basic set of CM fonts.

\newif\iflucida
\newif\ifbasiccm
\newcount\fontscheme

% Say `\def\fonts{2}\input book' to get CMR
\ifx\fonts\undefined\else\fontscheme=\fonts\fi

\ifnum\fontscheme<2 \lucidatrue\else\lucidafalse\fi
\ifnum\fontscheme>3 \basiccmtrue \advance\fontscheme by-2 \else\basiccmfalse\fi

% To deal with the Sans font scheme (\fontscheme=1), we distinguish
% between the body font \tenrm (lbr or lsr as appropriate) and the
% math font \tenmrm (always lbr).  Same (and more important) for \tenit
% vs \tenmti and \tenbf vs \tenmbf.
%
% So there are three potentially different italic fonts: the italic
% used for running text (\tenit), the text italic font that is used for
% maths (\tenmti), and the math italic font (\teni).  There are two
% font-changing commands: 
%
%     Command	Horiz mode	Math mode
%
%     \it	\tenit		\tenmti
%     \mti	\tenmti		(no effect)
% 
% In math mode, you get \teni by default with no font-changing command.

% Font substitution for basic CM

\def\fontsubst#1#2{\expandafter\def\csname subst@#1\endcsname{#2}}

\ifbasiccm
  \fontsubst{cmcsc8}{cmcsc10 at8pt}
  \fontsubst{cmcsci8}{cmcsc10 at8pt}
  \fontsubst{cmssbx8}{cmssbx10 at8pt}
  \fontsubst{cmitt8}{cmitt10 at8pt}

  \fontsubst{cmcsc9}{cmcsc10 at9pt}
  \fontsubst{cmcsci9}{cmcsc10 at9pt}
  \fontsubst{cmssbx9}{cmssbx10 at9pt}
  \fontsubst{cmitt9}{cmitt10 at9pt}

  \fontsubst{cmcsci10}{cmcsc10}

  \fontsubst{cmcsci12}{cmcsc10 scaled\magstep1}
  \fontsubst{cmitt12}{cmitt10 scaled\magstep1}
  \fontsubst{cmsy12}{cmsy10 scaled\magstep1}
  \fontsubst{cmcsc12}{cmcsc10 scaled\magstep1}
  \fontsubst{cmex12}{cmex10 scaled\magstep1}
  \fontsubst{cmssbx12}{cmssbx10 scaled\magstep1}

  \fontsubst{cmr14}{cmr10 scaled\magstep2}
  \fontsubst{cmss14}{cmss10 scaled\magstep2}

  \fontsubst{cmr20}{cmr10 scaled\magstep4}
  \fontsubst{cmss20}{cmss10 scaled\magstep4}
  \fontsubst{cmssi20}{cmssi10 scaled\magstep4}
  \fontsubst{cmssbx20}{cmssbx10 scaled\magstep4}

  \fontsubst{cmss24}{cmss10 scaled\magstep5}
  \fontsubst{cmssbx24}{cmssbx10 scaled\magstep5}
\fi

% Use e.g. \@font\twelverm{cmcsci10 scaled\magstep1} to load a font,
% but with a possible substitution cmcsci10 --> cmcsc10, giving the
% effect of \font\twelverm=cmcsc10 scaled\magstep1
\def\@font#1#2{
  \expandafter\let\expandafter\next
    \csname subst@\expandafter\killscale#2 scaled\killscale\endcsname
  \ifx\next\relax \font#1=#2 
  \else \expandafter\getscale#2scaledscaled\getscale 
    \font#1=\next\scale \fi}

\def\killscale#1 scaled#2\killscale{#1}
\def\getscale#1scaled#2scaled#3\getscale{\def\scale{#2}%
  \ifx\scale\empty\else \def\scale{ scaled #2}\fi}


%
% Size selection
%

\newdimen\fontquad 
\newdimen\fontbody
\newdimen\blockindent

\newfam\sffam
\newfam\arfam % used for arrow font in Lucida

\newif\ifleaded
\iflucida\leadedtrue\fi

\newtoks\sizetoks

% Can set running text in 8, 9, 10, 11 or 12 point

\def\setsize#1#2#3#4#5#6{%
% #1 = text size
% #2 = script size
% #3 = scriptscript size
% #4 = quad and body size (CM)
% #5 = body size (Lucida)
% #6 = jot
  \ifleaded \fontbody=#5 \else \fontbody=#4 \fi
  \fontquad=#4 \jot=#6
  \parindent=\fontquad
  \blockindent=2\fontquad
  \def\@textsize{#1}\def\@scriptsize{#2}\def\@scriptscriptsize{#3}%
  	% NB: these sizes are used by the math quotes "..."
  \setfamily0{mrm} \setmface0{rm} \setface{mrm}
  \setfamily1{i}
  \setfamily2{sy}
  \setexfamily
  % The remaining families are not to be used in subscripts
  \smallfamily\itfam{mti} \setmface\itfam{it} \setface{mti} 
  \smallfamily\bffam{mbf} \setmface\bffam{bf} \setface{mbf} 
  \smallfamily\sffam{sf}  \setmface\sffam{sf} 
  \iflucida\smallfamily\arfam{ar}\fi
  \the\sizetoks
  \normalbaselines\rm}

\def\setface#1{\expandafter\edef\csname p#1\endcsname{%
  \expandafter\noexpand\csname\@textsize#1\endcsname}\ignorespaces}
\def\setmface#1#2{\expandafter\edef\csname p#2\endcsname{%
  \fam#1\expandafter\noexpand\csname\@textsize#2\endcsname}\ignorespaces}
\def\setfamily#1#2{%
  \textfont#1=\csname\@textsize#2\endcsname
  \scriptfont#1=\csname\@scriptsize#2\endcsname
  \scriptscriptfont#1=\csname\@scriptscriptsize#2\endcsname\ignorespaces}
\def\smallfamily#1#2{%
  \textfont#1=\csname\@textsize#2\endcsname\ignorespaces}
\def\setexfamily{%
  \textfont3=\csname\@textsize ex\endcsname
  \scriptfont3=\csname\@textsize ex\endcsname
  \scriptscriptfont3=\csname\@textsize ex\endcsname}

\def\eightpoint{\setsize{eight}{six}{five}{10pt}{11pt}{2pt}}
\def\ninepoint{\setsize{nine}{six}{five}{11pt}{12pt}{2.5pt}}
\def\tenpoint{\setsize{ten}{seven}{five}{12pt}{13pt}{3pt}}
\def\elevenpoint{\setsize{eleven}{eight}{six}{13pt}{14pt}{3.5pt}}
\def\twelvepoint{\setsize{twelve}{eight}{six}{14pt}{15pt}{3.5pt}}

\def\deffont#1{%
  % Add definition of face to \sizetoks
  \expandafter\sizetoks\expandafter{\the\sizetoks\setface{#1}}
  % Define the font-changing command
  \expandafter\edef\csname #1\endcsname{\protect\expandafter
    \noexpand\csname p#1\endcsname}}

\def\rm{\protect\prm} \def\mrm{\protect\pmrm} \def\it{\protect\pit}
\def\mti{\protect\pmti} \def\bf{\protect\pbf} \def\mbf{\protect\pmbf}
\def\sf{\protect\psf}

\deffont{sl} \deffont{sc} \deffont{sci} \deffont{sfb} \deffont{vsf}
\deffont{vsfi} \deffont{tt} \deffont{tti} \deffont{kr} \deffont{ki}

\def\normalbaselines{%
  \lineskip=\normallineskip
  \baselineskip=\fontbody
  \lineskiplimit=\normallineskiplimit
  \bigskipamount=\fontbody plus5pt minus3pt
  \medskipamount=0.5\fontbody plus4pt minus2pt
  \abovedisplayskip=\medskipamount
  \belowdisplayskip=\medskipamount
  \abovedisplayshortskip=\abovedisplayskip
  \belowdisplayshortskip=\belowdisplayskip
  \setbox\strutbox=\hbox{\vrule height0.7\fontbody depth0.3\fontbody width0pt}}

% We do not have oldstyle digits.
\let\oldstyle=\undefined

% \sl is not used for math setting
\textfont\slfam=\nullfont \scriptfont\slfam=\nullfont
  \scriptscriptfont\slfam=\nullfont


%
% Loading of fonts
%

\def\smallsizes{
  \do{five}{5.2}{5}{5}
  \do{six}{6.1}{6}{6}
  \do{seven}{6.9}{7}{7}}

\def\bigsizes{
  \do{eight}{7.8}{8}{8}
  \do{nine}{8.6}{9}{9}
  \do{ten}{9.5}{10}{10}
  \do{eleven}{10.4}{11}{10 scaled\magstephalf}
  \do{twelve}{11.2}{12}{12}}

\def\allsizes{\smallsizes\bigsizes}

\def\loadfont#1#2#3#4#5{
  \def\do##1##2##3##4{% name lusize cmsize basiccmsize
    \edef\f@nt{\ifcase\fontscheme #2 at##2pt\or #3 at##2pt\or
      \ifbasiccm #4##4\else #4##3\fi\or \ifbasiccm #5##4\else #5##3\fi\fi}
    \expandafter\@font\csname ##1#1\endcsname\f@nt}}

\def\specialfont#1#2#3{%
  \ifcase\fontscheme \@font#1{#2} \or\@font#1{#2} 
    \or\@font#1{#3} \or\@font#1{#3}\fi}

% Extra sizes for Lucida
\def\lfourteen{13pt}
\def\ltwenty{18pt}
\def\ltwentyfour{22pt}

\loadfont{mrm}  {lbr}    {lbr}    {cmr}    {cmr}    \allsizes
\loadfont{i}    {lbmi}   {lbmi}   {cmmi}   {cmmi}   \allsizes
\loadfont{sy}   {lbms}   {lbms}   {cmsy}   {cmsy}   \allsizes
\iflucida
  \loadfont{ar} {lbma}   {lbma}   {***}    {***}    \allsizes
\fi

\loadfont{rm}   {lbr}    {lsr}    {cmr}    {cmss}   \bigsizes
\loadfont{it}   {lbi}	 {lsi}    {cmti}   {cmssi}  \bigsizes
\loadfont{ex}   {lbme}   {lbme}   {cmex}   {cmex}   \bigsizes
\loadfont{bf}   {lbd}    {lsd}    {cmbx}   {cmssbx} \bigsizes
\loadfont{sf}   {lsr}    {lsr}    {cmss}   {cmss}   \bigsizes
\loadfont{sfi}	{lsi}	 {lsi}	  {cmssi}  {cmssi}  \bigsizes
\loadfont{mti}	{lbi}    {lbi}    {cmti}   {cmti}   \bigsizes
\loadfont{mbf}	{lbd}    {lbd}    {cmbx}   {cmbx}   \bigsizes
\loadfont{sl}	{lbsl}   {lso}    {cmsl}   {cmssi}  \bigsizes
\loadfont{sc}	{lbrsc}  {lbrsc}  {cmcsc}  {cmcsc}  \bigsizes
\loadfont{sci}	{lbosc}  {lbosc}  {cmcsci} {cmcsci} \bigsizes
\loadfont{sfb}	{lsd}    {lsd}    {cmssbx} {cmssbx} \bigsizes
\loadfont{vsf}	{vlsr}   {lbtn}   {cmtt}   {cmtt}   \bigsizes
\loadfont{vsfi}	{vlsi}   {lbtno}  {cmitt}  {cmitt}  \bigsizes
\loadfont{tt}	{lbtn}   {lbtn}   {cmtt}   {cmtt}   \bigsizes
\loadfont{tti}	{lbtno}  {lbtno}  {cmitt}  {cmitt}  \bigsizes
\loadfont{kr}   {lbkr}   {lbkr}   {cmss}   {cmss}   \bigsizes
\loadfont{ki}   {lbki}   {lbki}   {cmssi}  {cmssi}  \bigsizes

% Set the skewchars
\def\do#1#2#3#4{
  \expandafter\skewchar\csname #1i\endcsname='177
  \expandafter\skewchar\csname #1sy\endcsname='60
}
\allsizes

% Special fonts

% Fourteen point
\specialfont\fourteenrm{lbr at\lfourteen}{cmr14}
\specialfont\fourteensf{lsr at\lfourteen}{cmss14}

% Twenty point
\specialfont\twentyrm{lbr at\ltwenty}{cmr20}
\specialfont\twentysf{lsr at\ltwenty}{cmss20}
\specialfont\twentysfb{lsb at\ltwenty}{cmssbx20}
\specialfont\twentysfi{lsi at\ltwenty}{cmssi20}

% Twenty-four point
\specialfont\twentyfoursf{lsr at\ltwentyfour}{cmss24}
\specialfont\twentyfoursfb{lsd at\ltwentyfour}{cmssbx24}
\specialfont\twentyfourfunky{lbki at\ltwentyfour}{cmssbx24}

% Typeface loads a single face in multiple sizes.  The Lucida-adjusted
% size is used in each case, even if the font scheme is CM.
% Best used to load funky variants of Lucida
\def\typeface#1#2{\choosefont{#1}{#2}\bigsizes \deffont{#1}}

\def\choosefont#1#2{
  \def\do##1##2##3##4{% name lusize cmsize basiccmsize
    \expandafter\@font\csname ##1#1\endcsname{#2 at##2pt}}}

%
% Adjustments for Lucida
%

\iflucida

\def\neq{\mathrel{\fam\arfam\mathchar"7094}}

% Adjusted for LucidaNewMath-Extension at 9.5pt and math axis at 3.13pt
% Note: delimiter increments are 5.5pt (as opposed to 6pt in CM)
% These work in \tenpoint size only.

\def\big#1{{\hbox{$\left#1\vbox to8.20\p@{}\right.\n@space$}}}
\def\Big#1{{\hbox{$\left#1\vbox to10.80\p@{}\right.\n@space$}}}
\def\bigg#1{{\hbox{$\left#1\vbox to13.42\p@{}\right.\n@space$}}}
\def\Bigg#1{{\hbox{$\left#1\vbox to16.03\p@{}\right.\n@space$}}}
\def\biggg#1{{\hbox{$\left#1\vbox to17.72\p@{}\right.\n@space$}}}
\def\Biggg#1{{\hbox{$\left#1\vbox to21.25\p@{}\right.\n@space$}}}
\def\n@space{\nulldelimiterspace\z@ \m@th}

% define some extra large sizes - always done using extensible parts

\def\bigggl{\mathopen\biggg}
\def\bigggr{\mathclose\biggg}
\def\Bigggl{\mathopen\Biggg}
\def\Bigggr{\mathclose\Biggg}

%% %  Following is needed if the roman text font is NOT just LucidaBright
%% 
%% %  Draw the small sizes of `[' and `]' from LBMO instead of LBR
%% 
%% \mathcode`\[="4186 \delcode`\[="186302
%% \mathcode`\]="5187 \delcode`\]="187303
%% 
%% %  Draw the small sizes of `(' and `)' from LBMO instead of LBR
%% 
%% \mathcode`\(="4184 \delcode`\(="184300
%% \mathcode`\)="5185 \delcode`\)="185301
%% 
%% %  The small sizes of `{' and `}' are already drawn from LBMS instead of LBR
%% 
%% %  Draw small `/' from LBMO instead of LBR
%% 
%% \mathcode`\/="013D \delcode`\/="13D30E
%% 
%% %  Draw  `=' and `+' from LBMS instead of LBR
%% 
%% \mathcode`\=="3283 \mathcode`\+="2282
%% 
%% % May want to comment out this last one if text font IS known to be LBR

% plain.tex draws upper case upright Greek from roman text font ---
% --- need to draw instead from LucidaNewMath-Extension

\mathchardef\Gamma="03D0
\mathchardef\Delta="03D1
\mathchardef\Theta="03D2
\mathchardef\Lambda="03D3
\mathchardef\Xi="03D4
\mathchardef\Pi="03D5
\mathchardef\Sigma="03D6
\mathchardef\Upsilon="03D7
\mathchardef\Phi="03D8
\mathchardef\Psi="03D9
\mathchardef\Omega="03DA

% The dot accent has moved
\def\dot{\mathaccent"7005 }

\def\TeX{T\kern-.18em\lower.4ex\hbox{E}\kern-.1emX}

\def\copyright{{\rm\char"A9}}

\def\matrix#1{\null\,\vcenter{\normalbaselines\m@th
    \ialign{\hfil$##$\hfil&&\quad\hfil$##$\hfil\crcr
      \mathstrut\crcr\noalign{\kern-0.9\baselineskip}
      #1\crcr\mathstrut\crcr\noalign{\kern-0.9\baselineskip}}}\,}

% \p@renwd is used in \bordermatrix as the width of big extensible left paren
% i.e. where rows and columns are labelled outside parenthesized matrix
% \newdimen\p@renwd
\setbox\z@=\hbox{\tenex B} \p@renwd=\wd\z@ % width of the big left (

% following changed because `(' is not large enough for strut in LBMO
\def\mathstrut{\vphantom{f}}

% In n-th root, don't want the `n' to come too close to the radical
\def\r@@t#1#2{\setbox\z@\hbox{$\m@th#1\sqrt{#2}$}
  \dimen@\ht\z@ \advance\dimen@-\dp\z@
  \mkern5mu\raise.6\dimen@\copy\rootbox \mkern-7.5mu \box\z@}

\fi


%
% Math stuff
%

\def~{\relax\ifmmode\,\else\penalty10000\ \fi}
\def\{{\relax\ifmmode\lbrace\else\iflucida\char`\{\else$\lbrace$\fi\fi}
\def\}{\relax\ifmmode\rbrace\else\iflucida\char`\}\else$\rbrace$\fi\fi}

% ! is an Ord, not a Close as in plain TeX
\mathcode`\!="0021

% Math spacing is a bit more chunky that in plain TeX
\thinmuskip=4mu plus2mu minus1mu
\medmuskip=5mu plus2mu minus2mu
\thickmuskip=6mu plus3mu

% In cmr10 and plain TeX, an ordinary space is 12/36 plus 6/36 minus 4/36 em,
% 1 mu is 2/36 em, and the three mu skips are
%	thin  3 mu 			= 6/36 em
%	med   4 plus 2 minus 4 mu	= 8/36 plus 4/36 minus 8/36 em
%	thick 5 plus 5 mu		= 10/36 plus 10/36 em

% @ is a binary operator
\mathcode`\@="2040

% "stuff" sets stuff with math text italic.  We use an \hbox to deal
% with \_ nicely, and use \mathselect to get the appropriate size:
\def\kwote#1"{\mathselect\kw@te{#1}}
\def\kw@te#1#2{\hbox{\csname #1mti\endcsname #2\/}}
\mathcode`\"="8000
{\catcode`\"=\active \global\let"=\kwote}

% Like mathpalette, but the argument is a font size
\def\mathselect#1#2{\mathchoice{#1\@textsize{#2}}{#1\@textsize{#2}}%
  {#1\@scriptsize{#2}}{#1\@scriptscriptsize{#2}}}

\def\isc#1{\hbox{\sci \@cap#1\@cap\/}} % constant in caps and small caps
\def\@cap#1#2\@cap{#1\lowercase{#2}}

%\def\star{\mathord{*}}

\catcode`@=12

\tenpoint
